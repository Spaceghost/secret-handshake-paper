\documentclass[12pt]{article}
\usepackage{amsmath}               % great math stuff
\usepackage{amsfonts}              % for blackboard bold, etc
\usepackage{amsthm}                % better theorem environments
\usepackage{graphicx}

\newtheorem{proto}{Protocol}

\begin{document}
\title{Designing a Secret Handshake: Authenticated Key Exchange as a Capability System}
\author{Dominic Tarr}


\maketitle

\begin{abstract}
Capability Based Security is a conceptual framework for designing
decentralized access control systems, yet there is no widely
implemented protocol for establishing secure two-way communication
that also forms a capability system.
We examine the ways various key exchange protocols aren't
capability systems, and then present a secure key
exchange protocol designed with capability systems in mind.
In this protocol, the server's public key forms an access capability.
Using a preauthentication step, we authenticate the client before
the server, but still accomplish mutual authentication within 4 passes.
All long term keys are kept secret from any unauthenticated actors.

\end{abstract}

\section{Introduction}

The development of key exchange algorithms marked the dawn of
modern cryptography\cite{ndic}. Their development was motivated
by the desire for secure communications between two parties---yet
designing a practical and secure protocol for exchanging
a shared key between two authenticated parties is non-trivial\cite{aake}.

Much of the research into key exchange has produced whole ``families'' of
protocols \cite{sigma}. Protocols currently in widespread
use tend to be layered and configurable (TLS, SSH). This is not
to the benefit of application developers---gaining a sufficiently
nuanced understanding of such cryptosystems takes considerable study.
Providing the developer with \emph{more options} is to provide them
with \emph{more opportunities} to add critical security flaws to their
application. Thus, recent thought has steered towards providing simple constructions
with security properties that are easy to understand and use safely
\cite{nacl}. We apply this philosophy to the design of an authenticated
key exchange - a secret handshake. Since a key exchange can be designed
with so many possible properties, we adopt the framework of capability
systems\cite{unicap} and allow that to drive the design.
TahoeLAFS\cite{tahoe} is the inspiration for this decision.
We find that no currently available handshake protocol adequately meets
the needs of a capability system. Interestingly, a capability system
demands a higher degree of privacy than is provided by other available
protocols.

We will describe various key exchange protocols and examine why their
design does not form a capability system. We will discuss these
protocols abstractly, giving sufficient details to demonstrate their
cryptographic properties. In some cases, this means ignoring details
required for their actual implementation, for example the handling of
nonces. Finally, we will describe our capability based key exchange
protocol.

\section{Capability Systems}

In a capability system, the right to access a resource is granted
to the bearer of a \emph{capability} (cap). Capability
Based Security predates modern cryptography, but its concepts
map cleanly to the decentralized access control systems
enabled by cryptography. \footnote{
  There are limitations to a cryptographic capability system.
  It is possible to implement what \emph{Capability Myths Demolished}\cite{capmyths}
  describes as the \emph{Capabilities as Keys} model. In this model,
  actors may delegate via any channel they can write to.
  Contrast this with the \emph{Object Capability} model, which incorporates
  a strict type system that can distinguish between data and capabilities,
  and thus prevent capabilities from being mistakenly written as data.
}

For example, possessing the key to decrypt a file means you have
the \emph{capability} to \emph{read} that file. This can be implemented
in a decentralized fashion: An encrypted file is widely distributed
via a peer to peer network, but the read-cap is delegated
only to a trusted few.  Contrast this with an Access
Control List (ACL), the file is kept on a server, and only those
who can correctly authenticate can access the file: Like a bouncer
at a nightclub, a trusted gatekeeper is required to administer the
access list. A decentralized ACL is not possible.


\textsc{Delegation} is the act of granting one of your access
rights to another actor by sending them a capability.
In a well behaved capability system, there is no way to gain a
particular access right except through delegation.

\textsc{Unintended Delegation}. If you give your car keys to a valet
to park your car, that is delegation. If you leave your keys in your
car and it is stolen--that is an \emph{unintended delegation},
or a \emph{misdelegation}. It is not the same as if someone forcibly
gained entry to your car and hot-wired it. If someone ``steals'' your
car because you mistakenly gave them the keys, it is your own fault.
It is our responsibility as protocol designers to not build
misdelegations into the protocol itself.\footnote{
  AES is an example of a protocol that has been shown to contain a
  misdelegation\cite{aestiming}  (secure AES implementations are possible,
  but suffer from poor performance).
  A good protocol design  cannot prevent a malicious implementer,
  but it can  mitigate the effects of
  an incompetent one. Thus, a good protocol design at least prevents
  a malicious implementer from \emph{feigning incompetence}\cite{underhanded}.
  Thankfully, maliciousness is not as common as incompetence.
} If a system has unintended delegations at the protocol level,
it is not a well behaved capability system.

\textsc{Wildcard}. In some card games there is a designated ``wildcard''
that can be played as any other card. A token is called a \emph{wildcard}
if it enables a set of rights that should have to be delegated
individually.
I know of no cryptosystem that was intentionally designed to contain
wildcards. Presence of an unintended wildcard in a capability
system should be considered a design failure.

\textsc{Well Behaved}. A capability system is
\emph{well behaved} if it supports delegation but lacks unintended
delegations and wildcards.

\section{Notation}

$A \to B$ signifies a message passed from client to server, and
$A \gets B$ signifies a response. It's important to understand
at what point a party becomes authenticated, and at what point
that party knows it. If $A$ or $B$ is replaced with $?$ it
means that party is not yet authenticated, to the knowledge
of the receiver. If Alice receives a message $? \gets B$ it means
she now knows who Bob is, but Bob doesn't know who she is.
A handshake is not complete until both parties know who they are
talking to, and know the other party also knows. Thus a handshake
does not end until two passes marked $A \to B$ and $A \gets B$
are exchanged.

When a key pair is an argument to a function,
it's represented as a subscript. When a key is sent in a message,
only the public key is used, to be explicit I write that as $A_{p}$.
None of the protocols studied in this paper ever send a secret key
over the wire.

Long term keys for Alice and Bob are represented by $A,B$,
and ephemeral keys are $a,b$. $a \cdot b$ denotes a shared key derived
from $a$ and $b$. Again, when this is a function argument it is shown
as the subscript, $Box_{[a\cdot b]}(msg)$. Note that
$Box$ is authenticated encryption, it has the
properties of a mac as well as of encryption. $|$ denotes concatenation.

Finally, whenever a message is received it is immediately verified
and if invalid the handshake is aborted.

\section{Prior Art}

\subsection{Authenticated Key Exchange - STS, TLS, SSH}

Station to Station\cite{aake} is simple, and is the basis of several
popular protocols\footnote{
  TLS (when used with ciphersuites that contain DHE or ECDHE),
  and SSH share their basic design with STS, but using
  a hmac to prove they know $a\cdot b$ instead of encryption.
}.

\begin{samepage}
\begin{proto}[Station to Station]

  $$
  \begin{align*}
      ? \to \;?\;   &: a_{p} \\
      ? \gets B &: b_{p}, Box_{[a\cdot b]}(B_p | Sign_B(a_{p}|b_{p})) \\
      A \to B   &: Box_{[a\cdot b]}(A_{pub} | Sign_A(a_{p} | b_{p})) \\
      A \gets B &: Box_{[a\cdot b]}(okay) \\
  \end{align*}
  $$

  Alice sends a fresh ephemeral key to Bob, who creates one too,
  signs both keys and sends them back with his public key.
  Alice then boxes her public key and a signature to prove her identity.
  Finally, Bob boxes a standard message to show his acceptance\footnote{
    Often the description of a handshake protocol ends as soon
    as each party is authenticated, but before the client \emph{knows}
    it is authenticated. If this is the case, the client  could receive
    an authentication error (or dropped connection) at the application
    layer, which is awkward. Thus, they do not know they've been accepted
    until receiving the first encrypted message. For this reason
    I've presented STS as a 4 pass protocol, even though the original
    paper describes it as a 3 pass protocol.
  }.

\end{proto}
\end{samepage}

Neither party can be assured of the freshness of a message
unless it is a cryptographic reply to a known fresh value,
i.e. a value they freshly created. Hence, it is pointless for Alice
to send her identity in the first pass. Bob cannot be sure it truly her
until the third pass at the earliest.

However, Alice can know Bob's first message is fresh, so STS and many
other protocols send the server authentication as the second pass.
To resist an identity misbinding attack we require proof that the
other party possesses the shared secret $a\cdot b$. Authenticated encryption,
$Box_{[a\cdot b]}$, or a mac accomplishes that \cite[section 3.1]{sigma}


Is STS suitable for a capability system? No. The first thing STS
does is authenticate the server, but the first question a capability
system should ask is whether the client has the capability to
this resource. Since Alice needs to know $B_p$ to authenticate
Bob \emph{anyway}, it makes an excellent candidate for an access cap.

However, by authenticating Bob first, and sending his public key to
an unauthenticated client, the public keys are delegated to anyone
who initiates the protocol with Bob. Because of this unnecessary
delegation, STS is not very useful as a capability system\footnote{
  Readers familiar with SSH will know the typical workflow for setting
  up a new server and accessing it for the first time. A server is
  created (on a cloud service such as AWS), with the user's public
  key in the \texttt{authorized\_keys} file. The user connects with the
  ssh command, and since this is the first connection, they will be
  prompted to accept the server's \emph{key fingerprint}.
  Most users will just hit \texttt{Y}, since that is easier than
  checking. This allows a window for a man-in-the-middle attack,
  rendering ssh effectively a \emph{trust on first use} system.
  On the other hand, if SSH was a capability system, the user would
  need to copy/paste the cap, completely avoiding this weakness.
  The user is likely to copy/paste the IP address anyway, so if the
  address and key were combined into one token it would be no more
  awkward.
  SSH requires an inconvenient security discipline, only
  followed by the most paranoid users, but a capability system
  would provide naturally better security at the same level of
  convenience.
}.

\subsection{Encrypted Key Exchange - CurveCP}

CurveCP\cite{curvecp} implements authentication relying only on
nacl's\cite{nacl} $Box$ primitive, which uses curve25519 keys.
curve25519 keys are combined via scalar multiplication to produce
shared keys, like Diffie Helman keys\cite{ndic}, not signing keys.
CurveCP authenticates each party by showing they are able to produce
the shared key, and it uses nested boxes to protect that authentication
from eavesdroppers.

\begin{proto}[CurveCP]

  $$
  \begin{align*}
      ? \to B   &: a_{p}, Box_{[a\cdot B]}(okay) \\
      ? \gets B &: Box_{[a\cdot B]}(b_{p}) \\
      A \to B   &: Box_{[a\cdot b]}(A_p||Box_{[A \cdot B]}(a_{p}))\\
      A \gets B &: Box_{[a\cdot b]}(okay) \\
  \end{align*}
  $$

  Alice sends her ephemeral key, $a_p$, and a boxed standard message to Bob.
  Bob boxes his ephemeral key back to Alice's ephemeral key.
  This convinces Alice of Bob's identity, a man in the middle fails
  at this point. Alice boxes (to $[a\cdot b]$) her long term key, $A_p$,
  with another box (to $[A\cdot B]$) containing her ephemeral key.
  If Bob is satisfied with this he responds with a standard message,
  boxed to the ephemeral keys\footnote{
    Again, actually a 4 pass protocol.
  }.

\end{proto}

In CurveCP knowing Bob's public key functions as a capability
to access Bob. Unfortunately, this protocol has two problems.

\begin{enumerate}
  \item If Bob moves to a different address, a replay attack can
    confirm where, without knowing his public key.
    Since the first pass is encrypted only Bob will be able to respond,
    if it is replayed to a new server and that server responds,
    it  suggests that server is Bob.
    Having this information could encourage the replay attacker
    to look for other security weaknesses.

  \item Worse,
    step 3 fails to include a fresh value from both sides of the
    exchange. Thus $(a, Box_{[A \cdot B]}(a_{p}))$ forms a reusable capability
    to authenticate to Bob as Alice\cite{ccp_review}. See also
    \cite{aake}[section 5.1, signing only one's own exponential].

  \item Worst, CurveCP contains a wildcard capability.
    Because a curve25519 key is used for authentication,
    an actor who knows $B_{secret}$ \cite{ccp_review} can
    impersonate \emph{anyone} to Bob--In the literature
    this is called a Key Compromise Impersonation (KCI).
    If Conrad gains
    $B_{secret}$ he can impersonate \emph{anyone} to Bob.
    To impersonate Alice, Conrad would connect to Bob,
    create an ephemeral key $a$, box it to Bob, and when Bob responds,
    create $Box_{[a\cdot b]}(A_p||Box_{[A \cdot B]}(a_{p}))$,
    except using $(A_{public}, B_{secret})$,
    as Bob would when opening the box, not $(A_{secret}, B_{public})$ as
    Alice would when sealing it. If Conrad possesses $B_{secret}$ he can
    impersonate arbitrary keys to Bob. Thus, $B_{secret}$ is a wildcard
    capability, and CurveCP is not a well behaved capability system.

\end{enumerate}

\subsection{Deniable Key Exchange - OTR, Noise, TextSecure}

There is a class of key exchange protocols which make deniability
a design goal\cite{otr, textsecure, noise}. The argument for this
is that when you engage in casual communication you do not create
evidence that you said what you did.

These protocols are unsuitable for a capability system for the same
reason as STS---An unauthenticated client learns the server key in
the second pass. This would be simple to fix, so whether a deniable
key exchange can be a well behaved capability system deserves closer study.
TextSecure\cite{textsecure} supersedes OTR\cite{otr}, but
uses a 3rd party introducer and is thus not directly analogous to
a key exchange, so we will examine the noise\cite{noise} protocol
instead.

Here we will extend the notation for shared keys to express keys shared
between multiple pairs of keys. $a \cdot b | a \cdot B$ denotes
the hash of $a \cdot b$ concatenated with $a \cdot B$. Only a party
that can construct both the component keys can construct the composite
key.

\begin{proto}[noise]

  $$
  \begin{align*}
      ? \to \;?\; &: a_{p} \\
      ? \gets B &: b_{p}, Box_{[a\cdot b]}(B_{p}), Box_{[a\cdot b|a\cdot B]}(okay) \\
      A \to B &: Box_{[a\cdot b]}(A_{p}), Box_{[a\cdot b|A\cdot b]}(okay) \\
      A \gets B &: Box_{[a \cdot b | a \cdot B | A \cdot b]}(a_{p}|b_{p})\\
  \end{align*}
  $$

  Alice sends Bob her ephemeral key. Bob replies with his ephemeral key,
  a box containing his long term key, and a multi-keyed box as
  evidence he is Bob. Alice replies with
  a multi-key box as evidence she knows $a_{secret}$ and $A_{secret}$.
  Bob then sends back a boxed message showing he was also able
  to derive that shared secret\footnote{
    like STS, it's really a 4 pass protocol
  }

\end{proto}

Note that unlike CurveCP, a shared key is not derived between long term
 keys, but instead only between an ephemeral key and a long term key,
thus $B_{secret}$ is not a wildcard.

Even if Alice suspects that Conrad may have compromised her long term
key, $A$, she trusts that he surely cannot know her ephemeral key, $a$.
Without knowing $a_{secret}$ Conrad cannot construct $a\cdot B$, unless
it really \emph{is} Bob. To provide the same assurance to Bob,
they end up with the three way key $a \cdot b | a \cdot B | A \cdot b$,
as in TextSecure\cite{textsecure}. This seems reasonable,
except $b_{secret}$ is now a wildcard! (though, since it is ephemeral,
hopefully more difficult to gain)

\begin{enumerate}
  \item The handshake is protected from eavesdroppers -- but
  anyone who connects to the server will be sent the public key,
  so $B_p$ is not useful as a capability.

  \item Key Compromise Impersonation is still possible, but harder -- To impersonate an
  arbitrary key to Bob you have to know Bob's ephemeral \emph{and} long
  term secret keys. This would be possible for anyone who had
  passive read access to Bob's memory -- at first
  glance this may seem like a unlikely proposition, but in fact
  if Bob is running on a rented virtual machine that
  is precisely the situation he would be in.
  In a well behaved system, knowing Bob's secret key \emph{should}
  give you the ability to authenticate \emph{as Bob}, but not as Alice.
\end{enumerate}

Problem 2 would be avoided if the protocol was only ever run on
physical hardware---but this is unreasonable.
Since it still contains a wildcard this protocol
fails to form a well behaved capability system. Also, the property
of deniability feels difficult to reason about. Will the higher level
protocol introduce evidence of the communication? In any case, a secure
channel is used for a lot more than casual social communications,
and deniability does not appear to offer any special advantage
to a capability system.

\section{A New Design}

If we take some ideas from the noise and TextSecure handshakes,
but turn it around so that Alice authenticates first,
then we get something that starts to look like a capability system.

If Alice \emph{preauthenticates} Bob, then Bob can authenticate
Alice using one more pass. With two initial passes to prevent
replay attacks, we have a 4 pass protocol. This is no worse
than the above, even though we do not authenticate anyone
until the third pass.

To ``preauthenticate'', Alice sends a proof of both her identity,
and her intention to connect to Bob. Preauthentication can be
implemented with both encryption and signatures.

\begin{proto}[Deniable Capability Handshake]

  $$
  \begin{align*}
      ? \to \;?\; &: a_{p}  \\
      ? \gets \;?\; &: b_{p} \\
      A \to B &: Box_{[a \cdot b | a \cdot B]}(A_{p}) \\
      A \gets B &: Box_{[a \cdot b | a \cdot B | A \cdot b]}(okay)\\
  \end{align*}
  $$

  Alice sends her ephemeral key to Bob, who responds with his.
  Then Alice boxes her long term key to Bob so that only he can open it.
  Bob shows his acceptance by boxing a standard message
  so that only Alice or Bob can read it.

\end{proto}

Requiring Alice to authenticate first is unusual, but
I think this is a fair deal. Bob has already put himself
at a disadvantage by allowing himself to be publicly
addressable. It's only fair that Alice authenticates first.
By encrypting her authentication she need not reveal her
identity to anyone but the one true Bob. Likewise, if Bob chooses
not to accept the call, then Alice won't be able to deduce
whether or not it was really Bob. Maybe it was but he did not wish
to speak to her? Maybe it was just a wrong number? This protects
Bob from harassment.

If Bob is unconcerned with the identity of his clients,
he may allow anyone knowing $B_p$ to authenticate. A client can
remain anonymous by using a second ephemeral identity.

In this design Bob's public key is a capability,
but it still has the wildcard (KCI) problem that
noise and TextSecure have.

Key exchange is required for confidentiality and
forward security, but signatures are required to avoid wildcards.
With signatures, we'll have a truly well behaved capability system.

Since we will need both exchange and signing keys,
an identity could be represented by a pair of signing and exchange keys.
nacl uses ed25519 keys for signatures, and
curve25519 keys for exchange. However, nacl also provides
functions to convert signing to exchange keys,
so an identity could be represented as a signing key,
which would be converted to an exchange key when necessary.

\begin{samepage}
\begin{proto}[Capability-based Handshake]

  $$
  \begin{align*}
      ? \to ? &: a_{p}   \\
      ? \gets ? &: b_{p} \\
      H &= A_{p}|Sig_A(B_{p}|hash(a \cdot b)) \\
      A \to B &: Box_{[a \cdot b | a \cdot B]}(H)\\
      A \gets B &: Box_{[a \cdot b | a \cdot B | A \cdot b]}(Sig_B(H) )\\
  \end{align*}
  $$

  Alice and Bob exchange ephemeral keys, then Alice
  preauthenticates privately to Bob. Bob demonstrates his acceptance
  by privately signing her preauthentication.
  Note, that $a_{secret}$ and $b_{secret}$ are capabilities to decrypt
  the handshake, but since they are not wildcards we still have a well
  behaved capability system.

\end{proto}
\end{samepage}

The design is getting much better. We resist eavesdropping, replay,
man-in-the-middle, KCI attacks, and provide forward secrecy.
There are no wildcards, nor unintended delegations.
But there are still a few small issues to clear up.

Implementations change, and so it is often helpful to have a protocol
id and version associated with a handshake.
Also, if either of the two initial passes are tampered with, it would
be undetected until the first authenticated pass is received.
It might also be nice to prevent different applications
built on this protocol from interfering with each other.

Most importantly, we must be careful to avoid creating a vulnerability
to a Chosen Protocol Attack\cite{cpa}. If the signing keys are also used
elsewhere, it's possible that a signature from this protocol could be
reused, possibly creating an unintended delegation.
All of these issues can be addressed via an application key ($K$)
which is a capability to the protocol. The ephemeral keys can be
authenticated by using the $K$ as the key to an $hmac$. By including
$K$ in each signature it is demonstrated that the signature belongs
within this protocol, mitigating CPA. It is vital that if there are any
other cases where a signing key is used, then a similar level of care is
taken to ensure unambiguous interpretation of signatures.

$K$ is a capability to access \emph{an instance of this protocol}\footnote{
  i.e. an application which is built on it.
  $K$ should be updated if the implementation of the application changes
  incompatibly. If backwards compatibility is required, the new protocol
  version could be used on a new port, until the legacy version is fully
  deprecated. For an openly specified application $K$ may be publicly
  known. For a private application it may be a closely guarded secret.
}
An eavesdropper cannot extract $K$ from the cipher-text, but it can
confirm a correct guess at $K$. In the case where $K$ is widely known,
and an attacker that knows $K$ tampers with an initial pass but creates
avalid $hmac_K$, their interference will still be detected when the
third pass is received.

\begin{samepage}
\begin{proto}[Secret Handshake]

  $$
  \begin{align*}
      ? \to \;?\;   &: a_{p}, hmac_{K}(a_{p})   \\
      ? \gets \;?\; &: b_{p}, hmac_{[K|a\cdot b]}(b_{p}) \\
      H&=A_{p}|sign_A(K|B_{p}|hash(a\cdot b)) \\
      A \to B       &: box_{[K|a \cdot b | a \cdot B]}(H)\\
      A \gets B     &:
        box_{[K|a \cdot b | a \cdot B | A \cdot b]}(sign_B(K|H|hash(a\cdot b)) )\\
  \end{align*}
  $$

  The same as the previous, but the shared key starts
  as $K$, and is extended as more public keys are learned.
  Alice's authentication, $A_{p}|sign_A(K|B_p|hash(a\cdot b))$,
  proves that she possesses $A$, and that the proof is for this protocol
  (via $K$) and this handshake (via $hash(a\cdot b)$).
  In case Bob does not yet have Alice's key, we delegate it to him
  (obviously we want Bob to have that capability, or Alice would not have
  contacted him).
\end{proto}
\end{samepage}

For Bob to authenticate back to Alice, he could just sign the proof
Alice sent him, and send it back. $H$ is already cryptographically
linked to the preceding passes. On the other hand, it is easier to
persuade ourselves that signing $K$ mitigates CPA\cite{cpa} than that
$A_p$ is never the application key in another protocol.

Alice and Bob can now use their shared secret,
$K|a \cdot b|a \cdot B|A \cdot b$, with a bulk encryption protocol
to secure a two-way communication channel.

\section{Future Work}

The latency of a 4 pass protocol may be prohibitive for some
applications. A mechanism to prearrange a single-use
key for the next session may enable a two pass protocol, at least
once a given pair of actors have established contact.

Some readers will be wondering how Alice is to learn Bob's public key?
The conceptual framework of capability systems has a simple answer:
someone delegates it to her. In practice, there is a large design space
in the how and why.
Briefly, there could be a centralized registry, a DHT, a gossip network,
or access caps could be configured in files. Various combinations of
the above are likely, or different systems entirely.

\section{Conclusion}

I have described a highly private 4 pass handshake protocol
that is suitable for capability systems. It does not suffer
from replay, eavesdropping, man in the middle,
or key compromise impersonation.
Capability Systems provide a conceptual framework for both
distributing and restricting access, thus it is a helpful
guide when justifying decisions.
In cryptographic capability systems like this protocol, public and
private keys are simply \emph{access rights}. This allows us to
think otherwise unthinkable thoughts, such as the notion of a
\emph{secret public key} or a \emph{shared private key}.
With concepts like these, we can create dynamic
layers of access and restrictions. Since \emph{Secret Handshake} is
otherwise unopinionated, and secure two-way communication is fundamental,
if it was combined with a suitable streaming bulk encryption protocol,
it could be a powerful building block for decentralized access
control systems.

\section*{Acknowledgements}

I would like to thank Tristan Slominski, Paul Frazee, James Haliday,
and David Stainton for reading drafts of this paper and useful comments.

\bibliographystyle{plain}

\bibliography{shs}

\end{document}




