\documentclass[12pt]{article}
\usepackage{amsmath}               % great math stuff
\usepackage{amsfonts}              % for blackboard bold, etc
\usepackage{amsthm}                % better theorem environments

\begin{document}
\title{Designing a Secret Handshake: Authenticated Key Exchange as a Capability System}
\author{Dominic Tarr}


\maketitle

\begin{abstract}
  (abstract gets written last)
\end{abstract}

\section{Introduction}

The development of key exchange algorithms marked the dawn of
modern cryptography\cite{ndic}. Their development was motivated
by the desire for secure communications between two parties---yet
designing a practical and secure protocol for exchanging
a shared key between two authenticated parties is non-trivial\cite{aake}.

Much of the research into key exchange has produced whole ``families'' of
protocols \cite{sigma}. Protocols currently in widespread
use tend to be layered and configurable (TLS, SSH). This is not
to the benefit of application developers---gaining a sufficiently
nuanced understanding of such cryptosystems takes considerable study.
Providing the developer with \emph{more options} is to provide them
with \emph{more opportunities} to add critical security flaws to their
application. Thus, recent thought has steered towards providing simple constructions
with security properties that are easy to understand and use safely
\cite{nacl}. We apply this philosophy to the design of an authenticated
key exchange - a secret handshake. Since a key exchange can be designed
with so many possible properties, we adopt the framework of capability
systems\cite{unicap} and allow that to drive the design.

Cryptographic primitives such as a public or secret key are regarded
as read and write capabilities, respectively. TahoeLAFS\cite{tahoe} is
the inspiration for this decision. We find that no currently available
handshake protocol adequately meets the needs of a capability
system.
Interestingly, a capability system demands a higher degree
of privacy than is provided by other available protocols.

We will describe various key exchange protocols and examin why they
are not suitable for a capabality system. In the body of this paper
we will describe the protocols discussed abstractly, with sufficient
details to persuade the reader of their cryptographic protocols.
In some cases this will ignore details from their actual implementations.
Finally, we describe our capability based key exchange protocol.
In the appendix we have provided a description of our secret-handshake
with sufficient detail to produce a compatabile implementation.

\section{Notation}

$A \to B$ signifies a message from the client to the server, and
$A \gets B$ signifies a response. If $A$ or $B$ is replaced with $?$,
that indicates that party is not yet authenticated, to the knowledge
of the receiver. So for example, Bob sends a message to Alice proving
his identity but does not know who Alice is yet, that is represented
as $? \gets B$.

Long term keys are represented by capitals, and ephemeral keys are
represented in lower case. $a \cdot b$ denotes a shared key derived
from $a$ and $b$. When this is used as the key to an encryption,
it will be used in the subscript. $box_{[a\cdot b]}$. Note that
$box$ is authenticated encryption, so this construction has the
properties of a mac as well as of encryption. $|$ denotes concatenation.

Finally, it is implied that whenever an actor receives a signed,
encrypted or authenticated message they attempt to verify and
decrypt it immediately and if this fails then the handshake is immediately
aborted.

\section{Prior Art}

\subsection{Authenticated Key Exchange - Station to Station}

Here I will describe the STS protocol\cite{aake}, but this design also
forms the basis of most popular protocols (TLS, SSH) except
without encrypting the keys or signatures (remember that $box$
is authenticated so $box_k(msg)$ can be replaced with $mac_k(msg)$,
and it will still constitute a proof of possession of the secret $k$.

$$
\begin{align*}
\\
    ? \to \;?\;   &: a_p \\
    ? \gets B &: b_p, Box_{[a\cdot b]}(B_p | Sign_B(a_p|b_p)) \\
    A \to B   &: Box_{[a\cdot b]}(A_p | Sign_A(a_p | b_p)) \\
    A \gets B &: Box_{[a\cdot b]}(okay) \\
\\
\end{align*}
$$

Alice sends a fresh ephemeral key to Bob, who creates one too,
signs both keys and sends them back with his public key.
Neither party can be assured of the freshness of a message
unless it is a cryptographic response to a value they know is fresh,
i.e. that they just created. Hence is it no advantage for Alice to send
her identity in the first pass, Bob cannot be sure it truly her
until the third pass at the earliest.

However, Alice can know Bob's first message is fresh, so STS and many
other protocols send the server authentication in the second message.
To resist an identity misbinding attack we require proof that the
other party possesses the shared secret $a\cdot b$. Constructing
$box_{[a\cdot b]}$ or a mac accomplishes that \cite[section 3.1]{sigma}

Often in the description of a handshake protocol ends as soon
as it's possible for Alice and Bob to send each other a encrypted
message. However, if messages $A \to B$ and $A \gets B$ have not both
been received then at least one party does not know whether they are
authenticated. When this is the case, they do not know for sure
that they are authenticated until they receive the first encrypted
message in the main body of the communication session. For this reason
I've represented STS as a 4 pass protocol, although the original
paper describes it as 3 pass.

Is STS suitable for a capability system? No. The first thing STS
does is authenticate the server, but the first question a capability
system should ask is whether the client has the capability sufficient
to access this resource.

By authenticating Bob first, to an unauthenticated client, the public
keys are leaked to anyone who connects to Bob and executes the protocol
correctly.

\subsection{Encrypted Authenticated Key Exchange - CurveCP}

CurveCP\cite{curvecp} performs authentication based entirely on
nacl's\cite{nacl} $box$ primitive, which uses curve25519 keys.
curve25519 keys cannot be used to create signatures, but are combined
via scalar multiplication to produce shared keys. recall
$Box_{[a\cdot b]}(content)$ creates an authenticated, encrypted message
that can be opened by someone who knows either $(a_{secret}, b_{public})$,
or $(a_{public}, b_{secret})$. I have represented this with a $\cdot$
to remind the reader that this operation is undirected. A box is between
two keys, and not from one key to another -- unlike RSA encryption.

$$
\begin{align*}
\\
    ? \to \;?\;   &: a_p, Box_{[a\cdot B]}(okay) \\
    ? \gets B &: Box_{[a\cdot B]}(b_p) \\
    A \to B   &: Box_{[a\cdot b]}(A_p||Box_{[A \cdot B]}(a_p))\\
    A \gets B &: Box_{[a\cdot b]}(okay) \\
\\
\end{align*}
$$

CurveCP does have one feature that suggests it's suitability for
use in a capability system - for Alice to authenticate to Bob
she must already know his public key. Knowledge of the Bob's public
key thus forms a capability to connect to Bob.

Unfortunately, this construction has two problems.
\begin{enumerate}
\item because the first message is encrypted, a replay attacker may
  use it to confirm the identity of a server after it has moved
  to a different address, if it still uses the same key as before.
  Since only the owner of that key can decrypt that packet,
  if they respond it confirms they have the same identity.
  This reveals information that a replay attacker doesn't deserve.
  Having this information could motivate the replay attacker
  to look for other weaknesses in Bob, thus in the interest of
  easy to understand security properties this small leak must be closed.
  The simplest change to mitigate this attack to CurveCP would
  be to require the server to respond with nonsense to an incorrect
  initial packet, but this information would probably still be available
  via timing information and thus is contrary to the design philosophy
  of nacl\cite{nacl}

\item The simple way that CurveCP uses key sharing for authentication
  enables Key Compromise Impersonation by an attacker who gains possession
  of $B_{secret}$ \cite{ccp_review}. If Conrad gains
  $B_{secret}$ he can and impersonate Alice to Bob.
  Condrad would connect to Bob, create an ephemeral key $a$,
  box it to Bob, and when Bob responds, create
  $Box_{[a\cdot b]}(A_p||Box_{[A \cdot B]}(a_p))$, except using $(A_{public}, B_{secret})$,
  as Bob would when opening the box, not $(A_{secret}, B_{public})$ as Alice would
  have when sealing it. If Conrad possesses $B_{secret}$ he can impersonate arbitary
  keys to Bob. Authenticating as Alice in a capability system
  must only be possible via the possession of $A_{secret}$ - the
  capability to Alice's identity.
\end{enumerate}

\subsection{Deniable Authenticated Key Exchange - OTR, Noise, TextSecure}

There is a class of key exchange protocols which make deniability
a design goal\cite{otr, textsecure, noise}. The argument for this is that when you engage in casual
communication you do not create evidence that you said what you did.

The currently available protocols are unsuitable
for a capability system for the same reason as STS is unsuitable--
the transmission of Bob's long term key to an unauthenticated
client. This would be a simple fix, and whether a deniable key exchange
is suitable for a capability system deserves closer study.
Since TextSecure\cite{textsecure} supercedes OTR\cite{otr}, but
uses a 3rd party introducer and is thus not directly analogous to
a typical secure channel we will examine the noise\cite{noise} protocol
first.

Here we will extend the notation for shared keys to express keys shared
between multilpe pairs of keys. $a \cdot b | a \cdot B$ denotes
the hash of $a \cdot b$ concatenated with $a \cdot B$. Only a party
that can construct both the component keys can construct the composite
key.

$$
\begin{align*}
\\
    ? \to \;?\; &: a_p \\
    ? \gets B &: b_p, Box_{[a\cdot b]}(B_p), Box_{[a\cdot b|a\cdot B]}(okay) \\
    A \to B &: Box_{[a\cdot b]}(A_p), Box_{[a\cdot b|A\cdot b]}(okay) \\
    A \gets B &: Box_{[a \cdot b | a \cdot B | A \cdot b]}(a_p|b_p)\\
\\
\end{align*}
$$

This is a better design, but it has some features that make it less
suitable for a capability system. Note that although it is
specified as a 3 pass protocol the client does not know it is authorized
until it receives the first encrypted message from the server,
so again, it is really a 4 pass protocol. Note that unlike CurveCP,
a shared key is not derived between long term keys, but instead only
between an ephemeral key and a long term key.
Even if Alice suspects that Conrad may have compromised her long term
key, $A$, she trusts that he surely cannot know her ephemeral key, $a$.
Without knowing $a_{secret}$ Conrad cannot construct $a\cdot B$, unless
it really \emph{is} Bob. To provide the same assurance to Bob,
they end up with three way key $a \cdot b | a \cdot B | A \cdot b$,
as in TextSecure\cite{textsecure}.

\begin{enumerate}
  \item The handshake is protected from eavesdroppers -- but
  anyone who connects to the server will be sent the public key,
  so we cannot use knowledge of the server's key as a capability.
  \item KCI is still possible, but harder -- To impersonate an
  arbitary key to Bob you have to know Bob's ephemeral \emph{and} long
  term secret keys. This would be possible for anyone who had
  passive read access to Bob's memory -- at first
  glance this may seem like a unlikely proposition, but in fact
  if Bob is running on a rented virtual machine that
  is precisely the situation he would be in.
\end{enumerate}

Problem 2 would be avoided if the protocol was only ever run on
physical hardware--but this is completely unreasonable.
Since it's possible to authenticate to Bob as alice by knowing
$b$ and $B$, and not strictly by knowing $A$, then this protocol
fails to form a well behaved capability system. Also, the property
of deniability seems difficult to reason about. Will the higher level
protocol introduce evidence of the communication? Will the contents
stored for a long period of time? In any case, a secure channel is
used for a lot more than casual social communications, and deniability
does not appear to offer any special advantage to a capability system.

\section{A New Design}

If we take the general pattern from the noise handshake,
but turn it around so that Alice authenticates first,
then we start to get something that looks like a capability system.

If Alice \emph{preauthenticates} Bob, then Bob can authenticate
Alice using one more pass. With two initial passes to prevent
replay attacks, we have a 4 pass protocol. This is no worse
than the above, even though we do not authenticate anyone
until the third pass.

To ``preauthenticate'' Bob, Alice sends a proof of both her identity,
and her intention to connect to Bob. Preauthentication can be
implemented with both encryption and signatures.

$$
\begin{align*}
\\
    ? \to \;?\; &: a_p  \\
    ? \gets \;?\; &: b_p \\
    A \to B &: Box_{[a \cdot b | a \cdot B]}(A_p) \\
    A \gets B &: Box_{[a \cdot b | a \cdot B | A \cdot b]}(okay)\\
\\
\end{align*}
$$

Requiring Alice to authenticate first is unusual, but
I think this is a fair deal. Bob has already put themself
at a disadvantage by allowing himself to be publically
addressable. It's only fair that Alice authenticates first.
By encrypting her authentication she need not reveal her
identity to anyone but the one true Bob. Likewise, if Bob chooses
not to accept the call, then Alice won't be able to deduce
whether or not it was really Bob. Maybe it was but he did not wish
to speak to her, maybe it was just a wrong number. This protects
Bob from harassment.

In cases where Bob is not concerned with the identity of his clients,
he may simply allow anyone to authenticate, and Alice can
generate a second ephemeral identity instead of using $A$.

In this design Bob's public key forms as an access capability,
however it still suffers the KCI problem that noise and TextSecure have.

Authenticated handshakes based on signatures do not have these
KCI problems. Key exchange is required for confidentiality and
forward security, but signatures are required for simple resistance
to KCI. By incorporating a signatures into the handshake we can
achive a truly well behaved capability system.

Since we will need both exchange and signing keys,
an identity could be represented by a pair of signing and exchange keys.
nacl uses ed25519 keys for signatures, and
curve25519 keys for exchange. However, nacl also provides
a functions to convert signing to exchange keys,
so an identity could be represented as signing key.
This signing key would be converted to an exchange key when necessary.

$$
\begin{align*}
\\
    ? \to ? &: a_p   \\
    ? \gets ? &: b_p \\
    H &= A_{p}|Sig_A(B_p|hash(a \cdot b)) \\
    A \to B &: Box_{[a \cdot b | a \cdot B]}(H)\\
    A \gets B &: Box_{[a \cdot b | a \cdot B | A \cdot b]}(Sig_B(H) )\\
\\
\end{align*}
$$

The design is getting much better. We resist eavesdropping, replay,
man-in-the-middle, and KCI attacks. There are just a few minor
niggles to tidy up.

If either of the two initial packets are tampered with, it would
be undetected until the first authenticated packet is received,
this is not necessarily insecure, but it is poor design.
More importantly, if the signing keys are also used elsewhere,
it's possible that a signature from this protocol or gets
reused elsewhere, or vice versa \cite{cpa}. These issues can be
addressed by adding an application key ($K$) as a capability
to speak this protocol. The ephemeral keys can be authenticated by using
the $K$ as the key to an $hmac$. By including $K$ in each
signature it is demonstrated that the signature belongs within this
protocol. It is vital that if there are any other cases where a signing
key is used, then a similar level of care is taken to prevent ambigious
interpretations of signatures created.

$K_{app}$ could be the hash of the protocol version and the terms of service,
or it simply be a random number.

$$
\begin{align*}
\\
    ? \to \;?\;   &: a_p, hmac_{K}(a_p)   \\
    ? \gets \;?\; &: b_p, hmac_{[K|a\cdot b]}(b_p) \\
    H&=A_{p}|sign_A(K|B_p|hash(a\cdot b)) \\
    A \to B       &: box_{[K|a \cdot b | a \cdot B]}(H)\\
    A \gets B     &:
      box_{[K|a \cdot b | a \cdot B | A \cdot b]}(sign_B(K|H|hash(a\cdot b)) )\\
\\
\end{align*}
$$

Alice's authentication, $A_{p}|sign_A(K|B_p|hash(a\cdot b))$,
proves that she possesses $A$, and that this proof is for this protocol
(via $K$) and for this handshake (via $hash(a\cdot b)$).
Incase Bob does not have alice's key yet, she sends it with her public key.

For Bob to authenticate back to Alice, he could just sign the proof
Alice sent him, and send it back. $H$ is already cryptographically
linked to the preceding passes, however, it is easier to persuade
ourselves that signing $K$ mitigates the
Chosen Protocol Attack\cite{cpa} than that it's unlikely $A_p$
is never the access cap to another protocol.

Alice and Bob can now use their shared secret,
$K|a \cdot b|a \cdot B|A \cdot b$, with a bulk encryption protocol
to secure a two-way communication channel.

Usually a secret key represents an \emph{identity}, but one of the
interesting things in cryptographic capability systems, such as tahoeLAFS
is that a secret key more accurately a \emph{write capability}.
Systems built upon this secret handshake protocol may have shared
secret keys that allow other actors to authenticate to a particular
role semianonymously.

\section{Future Work}

The latency induced by a 4 pass protocol may be prohibitive for some
applications. If some mechanism was provided to prearrange a single
use key for the next session it may be possible to reduce this to two
passes, once a pair of actors have established contact.

Some readers will be wondering how Alice is to learn Bob's public key?
There is a large design space, from Certificate Authority style central
registries, to lookup via DHTs, to gossip networks or webs of trust.
Interestingly, although in this protocol capabilites are made from public
keys, they need not be ``public''. Access to any actor can be restricted
via a secret public key. Likewise, if an actor wishes to provide a
specific api on an anonymous basis, they may do so by creating a
shared private key--this is not an oxymoron in a capability system,
it is simply a capability that has been delegated. The use of such
capabilitys may be restricted to a limited number of times or within a
specific time window.

\section{Conclusion}

I have described a highly private 4 pass handshake protocol
that is suitable for capability systems. It does not suffer
from replay, eavesdropping, man in the middle,
or key compromise impersonation.
It can be described in just a few lines, and I have a reference
implementation which is just a few hundred.
Although the objective was a well-behaved capability system
it is interesting that this has produced a design with a higher
degree of privacy than other key exchange protocols.

\bibliographystyle{plain}

\bibliography{shs}

\end{document}




